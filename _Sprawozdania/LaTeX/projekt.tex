\documentclass[12pt]{article}
\usepackage[T1]{fontenc}
\usepackage[T1]{polski}
\usepackage[utf8]{inputenc}
\newcommand{\BibTeX}{{\sc Bib}\TeX} 
\usepackage{graphicx}
\usepackage{amsfonts}
\usepackage{subcaption}
\usepackage{float}


%%%%%%%%%%%%%%%%%%%%%%%%%%%%%%%%%%%%%%%% 
%%%%%%%%%%%%%%%%%%%% TRYB DRAFTU
\setkeys{Gin}{draft=True}
%%%%%%%%%%%%%%%%%%%%%%%%%%%%%%%%%%%%%%%%


\setlength{\textheight}{21cm}

\title{{\bf E-learningowy Projekt Zaliczeniowy}\linebreak
Eksploracja Danych Internetowych}
\author{Paweł Jeziorski, 234066 \and Karol Podlewski 234106}
\date{01.02.2021}

\begin{document}
\clearpage\maketitle
\thispagestyle{empty}
\newpage
\setcounter{page}{1}
\section{Cel zadania}

Celem zadania była implementacja sieci neuronowej typu autodekoder w celu kompresji obrazów.

\section{Wstęp teoretyczny}

Autodekoder to sieć neuronowa, która wykorzystywana jest przede wszystkim do ekstrakcji cech lub kompresji danych. Autodekoder to wielowarstwowy perceptron, składający się z neuronów liniowych, zawierający jedną warstwę ukrytą. Warstwy wejściowe oraz wyjściowe zawierają tyle samo neuronów, zaś warstwa ukryta zazwyczaj zawiera ich mniej - odpowiada ona za stratną kompresje danych.

\begin{figure}[H]
 \centering
 \includegraphics[width=0.8\linewidth]{projekt/autodekoder.png}
 \vspace{-0.1cm}
 \caption{Schemat sieci liniowej typu autodekoder \cite{instrukcja}}
 \label{FFT}
\end{figure}

W celu przetwarzania obrazów, należy pracować na ich pikselach. Obraz o wielkości 512x512 ładowany jest do pamięci właśnie w postaci tablicy dwuwymiarowej o takich właśnie rozmiarach. Żeby wprowadzić taką tablicę do sieci, należy ją najpierw spłaszczyć. Tworzenie wektora o 262 144 wejściach mija się z celem, dlatego operuje się na mniejszych fragmentach obrazu - np. w jednym momencie przetwarza się mniejszy fragment obrazu - kwadrat o wielkości 8x8 wymaga już tylko 64 neurony na wejściu oraz wyjściu.

Taki skompresowany obraz możliwy jest do oceny gołym okiem, ale jako dodatek przydatna może być także miara PSNR (ang. \textit{Peak Signal to Noise Ratio}), którą wylicza się ze wzoru

\begin{equation}
PSNR = 10 \cdot \log_{10} \left(\frac{255^2}{\frac{1}{512^2}\sum_{i=0}^{511}\sum_{j=0}^{511}(x_ij - \hat{x}_ij)^2} \right)
\end{equation}

gdzie mianownik ułamka w logarytmie wynika bezpośrednio z największej możliwej wartości do uzyskania, a iteracje do 512 elementów są powiązanie z wielkością przetwarzanego obrazu \cite{instrukcja}.

\subsection{Implementacja}

Zaimplementowana sieć neuronowa charakteryzuje się następującymi, zmiennymi parametrami:

\begin{itemize}
    \item \textbf{Liczba neuronów w warstwie ukrytej} - w ramach zadania zmieniana będzie liczba neuronów w warstwie ukrytej - wartości, jakie przetestowaliśmy to 2, 4, 6, 8, 10, 12, 20 oraz 30,
    \item \textbf{Liczba iteracji} - przetestowaliśmy algorytm dla 10000 iteracji.
    \item \textbf{Współczynnik uczenia} - w omawianym rozwiązaniu ustawiony na stałe na wartość 0,01,
    \item \textbf{Szerokość wzorca} - długość boku, który będzie poddawany uczeniu w sieci. W wypadku tego zadania jest to 8. Długość boku definiuje także liczbę neuronów w warstwie wejściowej oraz wyjściowej - jest to 64, czyli tyle samo, ile pikselów we fragmencie obrazu o wielkości 8x8, 
    \item \textbf{Liczba wzorców treningowych} - w przypadku omawianego rozwiązania zawsze jest to 10000.
\end{itemize}

Wszystkie funkcje aktywacji są funkcjami liniowymi, a stopień kompresji obrazu obliczany jest ze wzoru:

\begin{equation}
CR = \frac{bp \cdot wysokosc\_obrazu \cdot szerokosc\_obrazu }{bw \cdot ni \cdot nh + bh \cdot no + bw \cdot nh \cdot no}
\end{equation}

gdzie $bp$ to liczba bitów na piksel (w wypadku przetwarzanych obrazów jest to 8), $wysokosc$ i $szerokosc$ obrazu jest równa 512, $bw$ to bity potrzebne to zapamiętania wagi neuronu (we wzorze przyjęte zostało 8), $bh$ odpowiada bitom potrzebnym na zapamiętanie współczynnika na wyjściu warstwy ukrytej (przyjęte jako 12), a $ni$, $nh$ oraz $no$ to liczba neuronów w warstwach wejściowej, ukrytej i wyjściowej - na potrzebę realizacji tego zadania liczba $ni$ oraz $no$ jest stała i wynosi 64.


%%%%%%%%%%%%%%%%%%%%%%%%%%%%%%%%%%%%%%%%
%%%%%%%%%%%%%%%%%%%%   BADANIA

\newpage
\section{Eksperymenty i wyniki}

W ramach części doświadczalnej napisany program został sprawdzony na następujących 8 obrazach:



\begin{figure}[H]
\renewcommand*\thesubfigure{0\arabic{subfigure}} 
\centering
 \begin{subfigure}[b]{.24\linewidth}
   \centering
   \includegraphics[width=.9\textwidth]{projekt/01.png}
   \caption{}\label{fig:a_2}
 \end{subfigure}%  
  \begin{subfigure}[b]{.24\linewidth}
   \centering
   \includegraphics[width=.9\textwidth]{projekt/02.png}
   \caption{}\label{fig:a_4}
 \end{subfigure}%  
  \begin{subfigure}[b]{.24\linewidth}
   \centering
   \includegraphics[width=.9\textwidth]{projekt/03.png}
   \caption{}\label{fig:a_6}
 \end{subfigure}%  
  \begin{subfigure}[b]{.24\linewidth}
   \centering
   \includegraphics[width=.9\textwidth]{projekt/04.png}
   \caption{}\label{fig:a_8}
 \end{subfigure}\\%  
 \begin{subfigure}[b]{.24\linewidth}
   \centering
   \includegraphics[width=.9\textwidth]{projekt/05.png}
   \caption{}\label{fig:a_10}
 \end{subfigure}%
  \begin{subfigure}[b]{.24\linewidth}
   \centering
   \includegraphics[width=.9\textwidth]{projekt/06.png}
   \caption{}\label{fig:a_12}
 \end{subfigure}%
  \begin{subfigure}[b]{.24\linewidth}
   \centering
   \includegraphics[width=.9\textwidth]{projekt/07.png}
   \caption{}\label{fig:a_20}
 \end{subfigure}%
  \begin{subfigure}[b]{.24\linewidth}
   \centering
   \includegraphics[width=.9\textwidth]{projekt/08.png}
   \caption{}\label{fig:a_30}
 \end{subfigure}\\%    
 \caption{Obrazy wykorzystane w części eksperymentalnej.}\label{obrazy}
\end{figure}

\subsection{Opis eksperymentów}

W ramach przeprowadzonych badań jedynym zmienianym czynnikiem były obrazy należące do grupy badawczej. Przeprowadzono 4 różne symulacje:

\begin{enumerate}
    \item Zbiór treningowy składający się z obrazów 01 oraz 03,
    \item Zbiór treningowy składający się z obrazów 02 oraz 04,
    \item Zbiór treningowy składający się z obrazów 05 oraz 06,
    \item Zbiór treningowy składający się z obrazów 06 oraz 08.
\end{enumerate}

Wyniki uzyskanych eksperymentów przedstawiono w formie wykresów PSNR do stopnia kompresji.

\newpage
\subsection{Wyniki}

Przykładowa kompresja obrazu dla zestawu 1 znajduje się na Rysunku \ref{kompresja}.

\begin{figure}[H]
\centering
 \begin{subfigure}[b]{.24\linewidth}
   \centering
   \includegraphics[width=.9\textwidth]{projekt/out_01_2.png}
   \caption{2 neurony}\label{fig:a}
 \end{subfigure}%  
  \begin{subfigure}[b]{.24\linewidth}
   \centering
   \includegraphics[width=.9\textwidth]{projekt/out_01_2.png}
   \caption{4 neurony}\label{fig:b}
 \end{subfigure}%  
  \begin{subfigure}[b]{.24\linewidth}
   \centering
   \includegraphics[width=.9\textwidth]{projekt/out_01_2.png}
   \caption{6 neuronów}\label{fig:c}
 \end{subfigure}%  
  \begin{subfigure}[b]{.24\linewidth}
   \centering
   \includegraphics[width=.9\textwidth]{projekt/out_01_2.png}
   \caption{8 neuronów}\label{fig:d}
 \end{subfigure}\\%  
 \begin{subfigure}[b]{.24\linewidth}
   \centering
   \includegraphics[width=.9\textwidth]{projekt/out_01_2.png}
   \caption{10 neuronów}\label{fig:e}
 \end{subfigure}%
  \begin{subfigure}[b]{.24\linewidth}
   \centering
   \includegraphics[width=.9\textwidth]{projekt/out_01_2.png}
   \caption{12 neuronów}\label{fig:f}
 \end{subfigure}%
  \begin{subfigure}[b]{.24\linewidth}
   \centering
   \includegraphics[width=.9\textwidth]{projekt/out_01_2.png}
   \caption{20 neuronów}\label{fig:g}
 \end{subfigure}%
  \begin{subfigure}[b]{.24\linewidth}
   \centering
   \includegraphics[width=.9\textwidth]{projekt/out_01_2.png}
   \caption{30 neuronów}\label{fig:h}
 \end{subfigure}\\%    
 \caption{Kompresja obrazu A dla zestawu treningowego 1} \label{kompresja}
\end{figure}

\begin{figure}[H]
 \centering
 \includegraphics[width=0.9\linewidth]{projekt/chart_01_03.png}
 \caption{Wykres dla pierwszego eksperymentu}
 \label{chart_01}
\end{figure}

\begin{figure}[H]
 \centering
 \includegraphics[width=0.9\linewidth]{projekt/chart_02_04.png}
 \caption{Wykres dla drugiego eksperymentu}
 \label{chart_02}
\end{figure}

\begin{figure}[H]
 \centering
 \includegraphics[width=0.9\linewidth]{projekt/chart_05_07.png}
 \caption{Wykres dla trzeciego eksperymentu}
 \label{chart_03}
\end{figure}

\begin{figure}[H]
 \centering
 \includegraphics[width=0.9\linewidth]{projekt/chart_06_08.png}
 \caption{Wykres dla czwartego eksperymentu}
 \label{chart_04}
\end{figure}



%%%%%%%%%%%%%%%%%%%%%%%%%%%%%%%%%%%%%%%%
%%%%%%%%%%%%%%%%%%%%   BIBLIOGRAFIA

\newpage
\section{Dyskusja i wnioski}

\subsection{Dyskusja}

Niezależnie od użytych obrazów do wytrenowania sieci, wszystkie z nich zachowywały się podobnie. Obrazy 02 i 05 zawsze cechowały się niską wartością PSNR (która w przypadku obrazu 05 przekroczyła 20 tylko dla najmniejszej kompresji przy treningu na niej samej). Najlepiej zachowywał się obraz 07, który potrafił przekroczyć wartość PSNR. Obrazy nigdy nie przekroczyły bariery 40 - zgadza się to z odczuciem wizualnym, gdyż zawsze można je odróżnić od oryginałów.

\subsection{Wnioski}

\begin{itemize}
    \item Wybór obrazów do nauki nie wpływa znacznie na jakość kompresji - nauka na danym obrazie wciąż rzadko skutkuje dużo lepszą kompresją pod względem jakości wynikowego obrazka.
    \item Algorytm w swojej podstawowej wersji potrzebuje wielu iteracji oraz wielu wzorców, by dobrze kompresować obrazy - prosty krok nauki 0,01 może się okazać niewystarczająco dobry przy bardziej skomplikowanych zadaniach, przydatne może się okazać także skorzystanie z innych właściwości MLP.
    \item Sieci neuronowe okazują się bardzo przydatne przy kompresji obrazów.
\end{itemize}


%%%%%%%%%%%%%%%%%%%%%%%%%%%%%%%%%%%%%%%%
%%%%%%%%%%%%%%%%%%%%   BIBLIOGRAFIA

\newpage
\renewcommand\refname{Bibliografia}
\bibliographystyle{plain}
\bibliography{projekt_bibliografia}

\end{document}
